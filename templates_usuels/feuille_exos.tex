\documentclass[12pt,a4paper]{article}

% Packages Ecriture
\usepackage[T1]{fontenc}
\usepackage[utf8]{inputenc}
\usepackage{fourier}
\usepackage[scaled=0.875]{helvet} 
\renewcommand{\ttdefault}{lmtt} 
\usepackage{eurosym}
\usepackage[french]{babel}
\usepackage[np]{numprint}

% Packages Mise en page
\usepackage{geometry} % Permet la mise en page
\geometry{lmargin=1cm,rmargin=1cm,tmargin=1cm,bmargin=1cm} % Choix format et  des marges
\usepackage{fancyhdr}
\usepackage{framed} % Création d'environnements
\usepackage[framed,thmmarks]{ntheorem} % Gestions des environnements theorem
\usepackage{lastpage}

\usepackage{amsmath,amssymb,makeidx}
\usepackage{enumerate}
\usepackage[normalem]{ulem}
\usepackage{fancybox,graphicx}
\usepackage{tabularx}
\usepackage{ulem}
\usepackage{dcolumn}
\usepackage{textcomp}
\usepackage{diagbox}
\usepackage{tabularx}
\usepackage{lscape}
\usepackage{pstricks,pst-plot,pst-text,pst-tree,pstricks-add,pst-grad,pst-coil,pst-blur}
\usepackage[pstricks]{bclogo} % Logo
%\setlength\paperheight{297mm}
%\setlength\paperwidth{210mm}
%\setlength{\textbfheight}{25cm}

\usepackage{multicol}

\usepackage{hyperref}

\usepackage{pgf,tikz,tkz-tab}
\usepackage{mathrsfs}
\usepackage{pifont}
\usetikzlibrary{arrows}

\usepackage{xcolor}
\usepackage{multicol}
\usepackage{multirow}


%%%%%%%%%%%%%%%%%%%%%%%%%%%%%%%%%%%%%%%%%%%%%%%%%%%%%%%%%%%%%%%%%%%%%%%%%%%%%%%%%%%%%%%%%%%%%%%%
%%%%%%%%%%%%%%%%%%%%%%%%%%%%%%%%%%%%%%%%%%%%%%%%%%%%%%%%%%%%%%%%%%%%%%%%%%%%%%%%%%%%%%%%%%%%%%%%

\DecimalMathComma	% Définit la virgule comme séparateur entre la partie entière et la partie décimale


\def\Oij{$\left(\textbfrm{O},~\vec{\imath},~\vec{\jmath}\right)$}
\def\Oijk{$\left(\textbfrm{O},~\vec{\imath},~\vec{\jmath},~\vec{k}\right)$}
\def\Ouv{$\left(\textbfrm{O},~\vec{u},~\vec{v}\right)$}

\newcommand{\vect}[1]{\mathchoice%
{\overrightarrow{\displaystyle\mathstrut#1\,\,}}%
{\overrightarrow{\textbfstyle\mathstrut#1\,\,}}%
{\overrightarrow{\scriptstyle\mathstrut#1\,\,}}%
{\overrightarrow{\scriptscriptstyle\mathstrut#1\,\,}}}

\newcommand{\equi}{\Leftrightarrow}
\newcommand{\pg}{\geqslant}
\newcommand{\pp}{\leqslant}
\newcommand{\dt}{\,\mathrm{d}t}
\newcommand{\dx}{\,\mathrm{d}x}

\newcommand{\N}{\mbox{${\mathbb N}$}}
\newcommand{\Z}{\mbox{${\mathbb Z}$}}
\newcommand{\Q}{\mbox{${\mathbb Q}$}}
\newcommand{\R}{\mbox{${\mathbb R}$}}
\newcommand{\C}{\mbox{${\mathbb C}$}}

%Commande pour créer des lignes de pointillés
\newcommand{\Pointilles}[1][3]{%
\multido{}{#1}{\makebox[\linewidth]{\dotfill}\\[\parskip]
}}

\DeclareMathOperator{\e}{e} %Permet d'écrire "droit" le e de l'exponentielle

\let\oldarray=\array
\def\array{\small\oldarray} % Permet d'écrire plus petit dans le tableau

\renewcommand{\arraystretch}{1.2} % Permet de gérer la hauteur des lignes des tableaux

\parindent=0mm % Supprime l'alinéa

\setlength{\fboxsep}{2mm} % Gère l'espacement entre un cadre et son contenu

\renewcommand{\thesection}{\Roman{section}.} % Numérote les sections en chiffre romain
\renewcommand{\thesubsection}{\Alph{subsection}.}

\newcounter{nexo}
\setcounter{nexo}{0}
\newcommand{\exo}{%
\stepcounter{nexo}
{\textbf{\underline{\textsc{Exercice \arabic{nexo}}}}%
\medskip%
}
}

\begin{document}

\setlength\parindent{0mm}

\fancyhf{\footnotesize{}}
\fancyhead[L]{\footnotesize{}}
\fancyhead[R]{\footnotesize{}}
\fancyfoot[L]{\small{\textbf{}}}
\fancyfoot[R]{\small{\textbf{}}}

%\cfoot{Page \thepage\ sur \pageref{LastPage}}

\renewcommand{\headrulewidth}{0pt}
\renewcommand{\footrulewidth}{0pt}

\renewcommand{\labelitemi}{$\bullet$}

\pagestyle{fancy}

%%%~\vspace{-2cm}

%%%\small{Mathématiques - Chapitre 1} \hfill \small{T\up{le}STI2D2}

\begin{center} {\large \framebox{\textbf{Feuille d'exercices 1 - Alignement de points}}} \end{center}

Cette feuille d'exercices est à traiter après avoir lu complètement le diaporama du cours. 

\medskip

\exo 

Dans le repère ci-dessous, on a tracé deux droites $d$ et $d'$. 

\begin{center}
\psset{unit=1.0cm,algebraic=true,dimen=middle,dotstyle=o,dotsize=5pt 0,linewidth=1.6pt,arrowsize=3pt 2,arrowinset=0.25}
\begin{pspicture*}(-5.,-5.)(5.,5.)
\multips(0,-5)(0,1.0){11}{\psline[linestyle=dashed,linecap=1,dash=1.5pt 1.5pt,linewidth=1pt,linecolor=lightgray]{c-c}(-5.,0)(5.,0)}
\multips(-5,0)(1.0,0){11}{\psline[linestyle=dashed,linecap=1,dash=1.5pt 1.5pt,linewidth=1pt,linecolor=lightgray]{c-c}(0,-5.)(0,5.)}
\psaxes[labelFontSize=\scriptstyle,xAxis=true,yAxis=true,Dx=1.,Dy=1.,ticksize=-2pt 0,subticks=2]{->}(0,0)(-5.,-5.)(5.,5.)
\psplot[linewidth=2.pt,linecolor=red]{-5.}{5.}{(--3.5-1.5*x)/1.}
\psplot[linewidth=2.pt,linecolor=blue]{-5.}{5.}{(--1.--0.6666666666666666*x)/1.}
\rput[bl](-3.5,-1.7){\blue{$d$}}
\rput[bl](3.5,-1.5){\red{$d'$}}
\end{pspicture*}
\end{center}

\begin{enumerate}
\item 
\begin{enumerate}
\item Déterminer, par lecture graphique, un vecteur directeur de la droite $d$ et un point $A$ appartenant à la droite $d$. \par 
En déduire une équation cartésienne de la droite $d$.

\item Le point $B(93 ; 63)$ appartient-il à la droite $d$ ? Et le point $C(-54 ; -35)$ ?
\item Que peut-on déduire pour les points $A$, $B$ et $C$ ?
\end{enumerate}

\item 
\begin{enumerate}
\item Déterminer l'équation réduite de la droite $d'$. 
\item Montrer que les points $D(-13;23)$ et $E(29 ; -40)$ appartiennent à la droite $d'$.
\item Soit le point $F(41 ; -60)$. Les points $D$, $E$ et $F$ sont-ils alignés ? Justifier.
\end{enumerate}
\end{enumerate}

\bigskip

\exo 

Dans chacune des questions suivantes, déterminer si les points $A$, $B$ et $C$ sont alignés. 

{\renewcommand{\theenumi}{\alph{enumi}}
\begin{enumerate}
\item $A(-6 ; 2)$, $B(1,1)$ et $C(4;-2)$ ;
\item $A(1 ; 4)$, $B(-1,-6)$ et $C(2;9)$ ;
%\item $A(-4 ; -1)$, $B(3,-2)$ et $C(10;-3)$ ;
%\item $A(-11 ; 0)$, $B(1,8)$ et $C(10;14)$ ;
\end{enumerate}}

\bigskip

\exo

Soient les points $A(-1;-2)$ et $B(1;4)$. 

\begin{enumerate}
\item Déterminer une équation cartésienne de la droite $(AB)$.
\item Le point $C(-3;-9)$ est-il aligné avec les points $A$ et $B$ ? Qu'en est-il du point $D(0;1)$ ?
\item Déterminer les réels $y_F$ et $x_G$ pour que les points $F(3;y_F)$ et $G(x_G ; -5)$ soient alignés avec $A$ et $B$.
\end{enumerate}

\bigskip

\end{document}

Pour tous les exercices, on considère un repère du plan. 

\exo 

Dans le repère donné ci-dessous, on a tracé trois droites $d_1$, $d_2$ et $d_3$. \par 
On a aussi représenté deux vecteurs $\overrightarrow{u}$ et $\overrightarrow{v}$.

\begin{center}
\newrgbcolor{qqzzqq}{0. 0.6 0.}
\newrgbcolor{ffwwqq}{1. 0.4 0.}
\newrgbcolor{qqzzff}{0. 0.6 1.}
\newrgbcolor{ududff}{0.30196078431372547 0.30196078431372547 1.}
\psset{unit=1.0cm,algebraic=true,dimen=middle,dotstyle=o,dotsize=5pt 0,linewidth=1.6pt,arrowsize=3pt 2,arrowinset=0.25}
\begin{pspicture*}(-4.5,-3.5)(5.5,4.5)
\multips(0,-3)(0,1.0){9}{\psline[linestyle=dashed,linecap=1,dash=1.5pt 1.5pt,linewidth=1pt,linecolor=lightgray]{c-c}(-4.5,0)(5.5,0)}
\multips(-4,0)(1.0,0){11}{\psline[linestyle=dashed,linecap=1,dash=1.5pt 1.5pt,linewidth=1pt,linecolor=lightgray]{c-c}(0,-3.5)(0,4.5)}
\psaxes[labelFontSize=\scriptstyle,xAxis=true,yAxis=true,Dx=1.,Dy=1.,ticksize=-2pt 0,subticks=2]{->}(0,0)(-4.5,-3.5)(5.5,4.5)
\psplot[linewidth=2.pt,linecolor=qqzzqq]{-4.5}{5.5}{(-4.--2.*x)/2.}
\psplot[linewidth=2.pt,linecolor=blue]{-4.5}{5.5}{(-1.-3.*x)/2.}
\psplot[linewidth=2.pt,linecolor=red]{-4.5}{5.5}{(--4.--2.*x)/4.}
\psline[linewidth=2.pt,linecolor=ffwwqq]{->}(2.,-2.)(5.,1.)
\psline[linewidth=2.pt,linecolor=qqzzff]{->}(1.,1.)(0.,4.)
\begin{scriptsize}
\rput[bl](3.35,-0.39){\ffwwqq{$\overrightarrow{u}$}}
\rput[bl](0.64,2.48){\qqzzff{$\overrightarrow{v}$}}
\rput[bl](4.66,2.28){\qqzzqq{$d_1$}}
\rput[bl](-2.7,2.7){\blue{$d_2$}}
\rput[bl](-3.68,-1.2){\red{$d_3$}}
\end{scriptsize}
\end{pspicture*}
\end{center}

\begin{enumerate}
\item Le vecteur $\overrightarrow{u}$ est-il un vecteur directeur de la droite $d_1$ ? de la droite $d_2$ ? Justifier.
\item Le vecteur $\overrightarrow{v}$ est-il un vecteur directeur de la droite $d_1$ ? de la droite $d_2$ ? Justifier.
\item Pour chaque droite, donner un autre vecteur directeur.  \textit{On donnera les coordonnées du vecteur.}
\end{enumerate}

\bigskip

\exo 

Dans chacune des questions suivantes, on donne les coordonnées de deux points $A$ et $B$ d'une droite $d$. On donne aussi les coordonnées d'un vecteur~$\overrightarrow{u}$. \par 
Dans chaque cas, déterminer, en justifiant, si le vecteur $\overrightarrow{u}$ est un vecteur directeur de la droite $d$.

\medskip

\textbf{a)~}$A(6 ; -4)$ ; $B( -2 ; 8)$ ; $\overrightarrow{u} \begin{pmatrix} -2 \\ 3 \end{pmatrix}$ ; \quad \textbf{b)~}$A(-1 ; 2)$ ; $B( 39 ; 17)$ ; $\overrightarrow{u} \begin{pmatrix} 10 \\ 7 \end{pmatrix}$ ; \quad \textbf{c)~}$A(-10 ; -3)$ ; $B(3 ; 18)$ ; $\overrightarrow{u} \begin{pmatrix} -117 \\ -189 \end{pmatrix}$.

%\begin{enumerate}
%\item $A(6 ; -4)$ ; $B( -2 ; 8)$ ; $\overrightarrow{u} \begin{pmatrix} -2 \\ 3 \end{pmatrix}$.
%\item $A(-1 ; 2)$ ; $B( 39 ; 17)$ ; $\overrightarrow{u} \begin{pmatrix} 10 \\ 7 \end{pmatrix}$.
%\item $A(-10 ; -3)$ ; $B(3 ; 18)$ ; $\overrightarrow{u} \begin{pmatrix} -117 \\ -189 \end{pmatrix}$.
%\end{enumerate}

\bigskip

\exo 

Dans chacune des questions suivantes, on donne les coordonnées d'un point $A$ et d'un vecteur directeur $\overrightarrow{u}$ d'une droite $d$ OU les coordonnées de deux points $A$ et $B$ d'une droite $d$. On donne aussi les coordonnées d'un point~$M$. \par 
Dans chaque cas, déterminer, en justifiant, si le point $M$ appartient à la droite $d$.

\medskip

{\small
\textbf{a)~} $A(-1 ; -2)$ ; $\overrightarrow{u} \begin{pmatrix} 1 \\ 1 \end{pmatrix}$ ; $M(11 ; 10)$ ;~~ \textbf{b)~} $A(-4 ; 3)$ ; $\overrightarrow{u} \begin{pmatrix} 6 \\ -5 \end{pmatrix}$ ; $M(2 ; 7)$ ;~~ \textbf{c)~} $A(5 ; 3)$ ; $B(2 ; -3)$ ; $M(-1 ; -9)$ ;~~ \textbf{d)~} $A(1 ; -4)$ ; $B(8 ; 0)$ ; $M(13 ; 5)$.
}

%\begin{enumerate}
%\item $A(-1 ; -2)$ ; $\overrightarrow{u} \begin{pmatrix} 1 \\ 1 \end{pmatrix}$ ; $M(11 ; 10)$.
%\item $A(-4 ; 3)$ ; $\overrightarrow{u} \begin{pmatrix} 6 \\ -5 \end{pmatrix}$ ; $M(2 ; 7)$.
%\item $A(5 ; 3)$ ; $B(2 ; -3)$ ; $M(-1 ; -9)$.
%\item $A(1 ; -4)$ ; $B(8 ; 0)$ ; $M(13 ; 5)$.
%\end{enumerate}

\bigskip

\exo 

%{\small
Dans un repère tracé sur votre cahier :

\begin{enumerate}
\item Tracer la droite $d_1$ passant par le point $A(1;1)$ et dont un vecteur directeur est $\overrightarrow{u} \begin{pmatrix} 3 \\ 3 \end{pmatrix}$.
\item Tracer la droite $d_2$ passant par le point $B(-1;3)$ et dont un vecteur directeur est $\overrightarrow{v} \begin{pmatrix} 4 \\ -2 \end{pmatrix}$.
\item Tracer la droite $d_3$ passant par le point $C(3;-0,5)$ et dont un vecteur directeur est $\overrightarrow{w} \begin{pmatrix} 5 \\ 3 \end{pmatrix}$.
\end{enumerate}
%}

\bigskip

\end{document}