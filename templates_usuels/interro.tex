\documentclass[12pt,a4paper]{article}

% Packages Ecriture
\usepackage[T1]{fontenc}
\usepackage[utf8]{inputenc}
\usepackage{fourier}
\usepackage[scaled=0.875]{helvet} 
\renewcommand{\ttdefault}{lmtt} 
\usepackage{eurosym}
\usepackage[frenchb]{babel}
\usepackage[np]{numprint}

% Packages Mise en page
\usepackage{geometry} % Permet la mise en page
\geometry{lmargin=2cm,rmargin=2cm,tmargin=1cm,bmargin=0.5cm} % Choix format et  des marges
\usepackage{fancyhdr}
\usepackage{framed} % Création d'environnements
\usepackage[framed,thmmarks]{ntheorem} % Gestions des environnements theorem
\usepackage{lastpage}

\usepackage{amsmath,amssymb,makeidx}
\usepackage{enumerate}
\usepackage[normalem]{ulem}
\usepackage{fancybox,graphicx}
\usepackage{tabularx}
\usepackage{ulem}
\usepackage{dcolumn}
\usepackage{textcomp}
\usepackage{diagbox}
\usepackage{tabularx}
\usepackage{lscape}
\usepackage{pstricks,pst-plot,pst-text,pst-tree,pstricks-add,pst-grad,pst-coil,pst-blur}
\usepackage[pstricks]{bclogo} % Logo
%\setlength\paperheight{297mm}
%\setlength\paperwidth{210mm}
%\setlength{\textheight}{25cm}

\usepackage{fancyhdr}
\usepackage{hyperref}

%\thispagestyle{empty}
%\usepackage[frenchb]{babel}
%\usepackage[np]{numprint}

\usepackage{pgf,tikz,tkz-tab}
\usepackage{mathrsfs}
\usetikzlibrary{arrows}

\usepackage{xcolor}


%%%%%%%%%%%%%%%%%%%%%%%%%%%%%%%%%%%%%%%%%%%%%%%%%%%%%%%%%%%%%%%%%%%%%%%%%%%%%%%%%%%%%%%%%%%%%%%%
%%%%%%%%%%%%%%%%%%%%%%%%%%%%%%%%%%%%%%%%%%%%%%%%%%%%%%%%%%%%%%%%%%%%%%%%%%%%%%%%%%%%%%%%%%%%%%%%

\DecimalMathComma	% Définit la virgule comme séparateur entre la partie entière et la partie décimale


\def\Oij{$\left(\textrm{O},~\vec{\imath},~\vec{\jmath}\right)$}
\def\Oijk{$\left(\textrm{O},~\vec{\imath},~\vec{\jmath},~\vec{k}\right)$}
\def\Ouv{$\left(\textrm{O},~\vec{u},~\vec{v}\right)$}

\newcommand{\vect}[1]{\mathchoice%
{\overrightarrow{\displaystyle\mathstrut#1\,\,}}%
{\overrightarrow{\textstyle\mathstrut#1\,\,}}%
{\overrightarrow{\scriptstyle\mathstrut#1\,\,}}%
{\overrightarrow{\scriptscriptstyle\mathstrut#1\,\,}}}

\newcommand{\equi}{\Leftrightarrow}
\newcommand{\pg}{\geqslant}
\newcommand{\pp}{\leqslant}
\newcommand{\dt}{\,\mathrm{d}t}
\newcommand{\dx}{\,\mathrm{d}x}

\newcommand{\N}{\mbox{${\mathbb N}$}}
\newcommand{\Z}{\mbox{${\mathbb Z}$}}
\newcommand{\Q}{\mbox{${\mathbb Q}$}}
\newcommand{\R}{\mbox{${\mathbb R}$}}
\newcommand{\C}{\mbox{${\mathbb C}$}}

%Commande pour créer des lignes de pointillés
\newcommand{\Pointilles}[1][3]{%
\multido{}{#1}{\makebox[\linewidth]{\dotfill}\\[\parskip]
}}

\DeclareMathOperator{\e}{e} %Permet d'écrire "droit" le e de l'exponentielle

\let\oldarray=\array
\def\array{\small\oldarray} % Permet d'écrire plus petit dans le tableau

\renewcommand{\arraystretch}{1.5} % Permet de gérer la hauteur des lignes des tableaux

\parindent=0mm % Supprime l'alinéa

\setlength{\fboxsep}{2mm} % Gère l'espacement entre un cadre et son contenu

\renewcommand{\thesection}{\Roman{section}.} % Numérote les sections en chiffre romain
\renewcommand{\thesubsection}{\Alph{subsection}.}


\newcounter{nexo}
\setcounter{nexo}{0}
\newcommand{\exo}{%
\stepcounter{nexo}
{\textbf{\underline{\textsc{Exercice \arabic{nexo}}}}%
\medskip%
}
}


\begin{document}

\setlength\parindent{0mm}
\rhead{DM6}
\lhead{\textsc{2GT2}}
%\rfoot{\small{Métropole--La Réunion}}
%\lfoot{\small{20 juin 2016}}
\renewcommand \footrulewidth{.2pt}
\pagestyle{empty}

\begin{center} \huge{\framebox{\textbf{Interrogation n°4 de Mathématiques - 2GT4}}} \end{center}

\begin{center} \Large{Lundi 16 décembre 2019} \end{center}

\medskip

Nom : \dotfill Prénom : \dotfill Note : \dotfill

\bigskip

\textit{Toutes les réponses devront être justifiées. Le barème est donné à titre indicatif.} \par 
\textbf{Les calculs doivent être détaillés.} \par
%\textbf{Le sujet est recto verso. N'oubliez pas de retourner la feuille !!!} \par

\medskip

Dans tous les exercices, on considère un repère $(O,I,J)$. 

\medskip

\exo \hfill (4 points)

On considère les points $A(5;10)$, $B(-4;8)$, $C(7;7)$ et $D(16;9)$.

\begin{enumerate}
\item On note $M$ le milieu du segment $[AC]$. Calculer les coordonnées du point $M$.

\vspace{4cm}

\item On note $N$ le milieu du segment $[BD]$. Calculer les coordonnées du point $N$.

\vspace{4cm}

\item À l'aide des questions précédentes, que peut-on dire du quadrilatère $ABCD$ ? Justifier. 

\vspace{3cm}

\end{enumerate}

\bigskip

\exo \hfill (2,5 points)

Soient les points $A(-6;11)$ et $M \left( -\dfrac{5}{2} ; 8 \right) $. On considère le point $B$, symétrique de $A$ par rapport à $M$. \par 
Déterminer, par le calcul et en justifiant complètement, les coordonnées du point $B$.

\bigskip

\newpage

~\vspace{5cm}

\exo \hfill (3,5 points)

On considère les points $R \left( -1 ; \dfrac{5}{3} \right)$, $S\left( -\dfrac{9}{2}  ; -5 \right)$ et $T\left( 3 ; -3 \right)$. Soit le point $U$ tel que le quadrilatère $RSTU$ soit un parallélogramme. \par 
Déterminer, par le calcul, les coordonnées du point $U$.

\begin{enumerate}
\item Calculer les coordonnées du milieu de $[RT]$. 

\vspace{6cm}

\item Déterminer, par le calcul et en expliquant la démarche, les coordonnées du point $U$.
\end{enumerate}

\bigskip


\newpage
\setcounter{nexo}{0}

\begin{center} \huge{\framebox{\textbf{Interrogation n°4 de Mathématiques - 2GT4}}} \end{center}

\begin{center} \Large{Lundi 16 décembre 2019} \end{center}

\medskip

Nom : \dotfill Prénom : \dotfill Note : \dotfill

\bigskip

\textit{Toutes les réponses devront être justifiées. Le barème est donné à titre indicatif.} \par 
\textbf{Les calculs doivent être détaillés.} \par
%\textbf{Le sujet est recto verso. N'oubliez pas de retourner la feuille !!!} \par

\medskip

Dans tous les exercices, on considère un repère $(O,I,J)$. 

\medskip

\exo \hfill (4 points)

On considère les points $A(4;-5)$, $B(20;-15)$, $C(10;-8)$ et $D(-6;6)$.

\begin{enumerate}
\item On note $M$ le milieu du segment $[AC]$. Calculer les coordonnées du point $M$.

\vspace{4cm}

\item On note $N$ le milieu du segment $[BD]$. Calculer les coordonnées du point $N$.

\vspace{4cm}

\item À l'aide des questions précédentes, que peut-on dire du quadrilatère $ABCD$ ? Justifier. 

\vspace{3cm}

\end{enumerate}

\bigskip

\exo \hfill (2,5 points)

Soient les points $A(5;-7)$ et $M \left( \dfrac{7}{2} ; 9 \right) $. On considère le point $B$, symétrique de $A$ par rapport à $M$. \par 
Déterminer, par le calcul et en justifiant complètement, les coordonnées du point $B$.

\bigskip

\newpage

~\vspace{5cm}

\exo \hfill (3,5 points)

On considère les points $R \left( 1 ; 3 \right)$, $S\left( 3  ; -\dfrac{5}{2} \right)$ et $T\left( -\dfrac{1}{3} ; 3 \right)$. Soit le point $U$ tel que le quadrilatère $RSTU$ soit un parallélogramme. \par 
Déterminer, par le calcul, les coordonnées du point $U$.

\begin{enumerate}
\item Calculer les coordonnées du milieu de $[RT]$. 

\vspace{6cm}

\item Déterminer, par le calcul et en expliquant la démarche, les coordonnées du point $U$.
\end{enumerate}

\bigskip

\end{document}