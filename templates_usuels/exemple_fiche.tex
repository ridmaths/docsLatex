\documentclass[12pt]{beamer}
\usetheme{Boadilla}
\usecolortheme{seahorse}
\usepackage[utf8]{inputenc}
\usepackage[french]{babel}
\usepackage[T1]{fontenc}
\usepackage{amsmath}
\usepackage{amsfonts}
\usepackage{amssymb}
\usepackage{mathrsfs}
\usepackage{graphicx}
\usepackage{lmodern}
\usepackage{tikz,tkz-tab}

\usepackage{pgf,pgfplots}
\pgfplotsset{compat=1.14}
\usetikzlibrary{arrows}

\usefonttheme[onlymath]{serif}

%\author{}
\title{Calculer des dérivées}
%\setbeamercovered{transparent} 
%\setbeamertemplate{navigation symbols}{} 
%\logo{} 
%\institute{} 
\date{} 
%\subject{} 
\begin{document}

\begin{frame}
\titlepage
\end{frame}

\begin{frame}
\tableofcontents
\end{frame}


\section{Tableau des dérivées usuelles}

\begin{frame}{Tableau récapitulatif des dérivées}
\begin{center}
\includegraphics[scale=0.7]{tableau_derivees.pdf}
\end{center}
\end{frame}

\section{Opérations sur les dérivées}

\subsection{Somme de deux fonctions}

\begin{frame}{Somme de deux fonctions - Propriété}
Soit $I$ un intervalle de $\mathbb{R}$ et $u$ et $v$ deux fonctions définies et dérivables sur $I$.

\medskip
\begin{exampleblock}{Propriété}
Soit la fonction $f$ définie sur $I$ par : $f(x) = u(x) + v(x)$. \par 
La fonction dérivée, notée $f'(x)$, est \[ f'(x) = u'(x) + v'(x). \]
\end{exampleblock}
\end{frame}

\begin{frame}{Somme de deux fonctions - Exemple}
\begin{block}{Exemple}
Soit la fonction $f$ définie sur $\mathbb{R}$ par $f(x) = x^2 + \cos(x)$. \par 
Calculer sa dérivée $f'(x)$.
\end{block}

\medskip

$f$ est définie comme la somme de deux fonctions. Donc :

\medskip

\begin{tabular}{rcl}
$f(x)$ & $=$ & $\underbrace{x^2 + \cos(x)}_{\textcolor{red}{u(x) + v(x)}}$ \\
\end{tabular}

\medskip

\begin{tabular}{rcl}
$f'(x)$ & $=$ & $\overbrace{2x + (-\sin(x))}^{\textcolor{red}{u'(x) + v'(x)}}$ \\
$f'(x)$ & $=$ & $2x -\sin(x)$
\end{tabular}

\medskip

Donc \fbox{$f'(x) = 2x-\sin(x)$}.

\end{frame}


\subsection{Produit d'un réel par une fonction}

\begin{frame}{Produit d'un réel par une fonction - Propriété}
Soit $I$ un intervalle de $\mathbb{R}$, $u$ une fonction définie et dérivable sur $I$ et $k$ un nombre réel. 

\medskip

\begin{exampleblock}{Propriété}
Soit la fonction $f$ définie sur $I$ par : $f(x) = k \times u(x)$. \par 
La fonction dérivée, notée $f'(x)$, est \[ f'(x) = k \times u'(x). \]
\end{exampleblock}
\end{frame}

\begin{frame}{Produit d'une réel par une fonction - Exemple}
\begin{block}{Exemple}
Soit la fonction $f$ définie sur $\mathbb{R}$ par $f(x) = 5x^3$. \par 
Calculer sa dérivée $f'(x)$.
\end{block}

\medskip

$f$ est définie comme le produit d'un nombre réel par une fonction. Donc :

\medskip

\begin{tabular}{rcl}
$f(x)$ & $=$ & $\underbrace{5x^3}_{\textcolor{red}{k \times u(x)}}$
\end{tabular}

\medskip

\begin{tabular}{rcl}
$f'(x)$ & $=$ & $\overbrace{5 \times 3x^2}^{\textcolor{blue}{k \times u'(x)}}$ \\
$f'(x)$ & $=$ & $15x^2$
\end{tabular}

\medskip

Donc \fbox{$f'(x) = 15x^2$}.

\end{frame}

\subsection{Produit de deux fonctions}

\begin{frame}{Produit de deux fonctions - Propriété}
Soit $I$ un intervalle de $\mathbb{R}$, $u$ et $v$ deux fonctions définies et dérivables sur $I$.

\medskip

\begin{exampleblock}{Propriété}
Soit la fonction $f$ définie sur $I$ par : $f(x) = u(x) \times v(x)$. \par 
La fonction dérivée, notée $f'(x)$, est \[ f'(x) = u'(x) \times v(x) + u(x) \times v'(x). \]
\end{exampleblock}
\end{frame}

\begin{frame}{Produit de deux fonctions - Exemple}
\begin{block}{Exemple}
Soit la fonction $f$ définie sur $\mathbb{R}$ par $f(x) = (2x+5)\sin(x)$. \par 
Calculer sa dérivée $f'(x)$.
\end{block}

\medskip

$f$ est définie comme le produit de deux fonctions. Donc :

\medskip

$f(x) = u(x) \times v(x)$, avec : 

\medskip

\begin{flushleft}
$u(x) = 2x+5$ ; \hspace{0.5cm} $u'(x) = 2$ ; \par
$v(x) = \sin(x)$ ; \hspace{0.5cm} $v'(x) = \cos(x)$.
\end{flushleft}

\medskip

\begin{tabular}{rcl}
$f'(x)$ & $=$ & $u'(x) \times v(x) + u(x) \times v'(x)$ \\
$f'(x)$ & $=$ & $2 \times \sin(x) + (2x+5) \times \cos(x)$
\end{tabular}

\medskip

Donc \fbox{$f'(x) =  2 \sin(x) + (2x+5) \cos(x)$}.

\end{frame}

\subsection{Quotient de deux fonctions}

\begin{frame}{Quotient de deux fonctions - Propriété}
Soit $I$ un intervalle de $\mathbb{R}$, $u$ une fonction définie et dérivable sur $I$ et $v$ une fonction définie et dérivable sur $I$ ne s'annulant pas sur $I$.

\medskip

\begin{exampleblock}{Propriété}
Soit la fonction $f$ définie sur $I$ par : $f(x) = \dfrac{u(x)}{v(x)}$. \par 
La fonction dérivée, notée $f'(x)$, est \[ f'(x) = \dfrac{u'(x) \times v(x) - u(x) \times v'(x)}{v(x)^2}. \]
\end{exampleblock}
\end{frame}

\begin{frame}{Quotient de deux fonctions - Exemple}
\begin{block}{Exemple}
Soit la fonction $f$ définie sur $\mathbb{R}$ par $f(x) = \dfrac{x^2+1}{\sin(x)}$. \par 
Calculer sa dérivée $f'(x)$.
\end{block}

\medskip

$f$ est définie comme le quotient de deux fonctions. Donc :

\medskip

$f(x) = \dfrac{u(x)}{v(x)})$, avec : 

\medskip

\begin{flushleft}
$u(x) = x^2+1$ ; \hspace{0.5cm} $u'(x) = 2x+1$ ; \par
$v(x) = \sin(x)$ ; \hspace{0.5cm} $v'(x) = \cos(x)$.
\end{flushleft}

\medskip

\end{frame}

\begin{frame}{Quotient de deux fonctions - Exemple - suite}
\begin{tabular}{rcl}
$f'(x)$ & $=$ & $\dfrac{u'(x) \times v(x) - u(x) \times v'(x)}{v(x)^2}$ \\
$f'(x)$ & $=$ & $\dfrac{(2x+1) \times \sin(x) - (x^2+1) \times \cos(x)}{(5x-12)^2}$.
\end{tabular}

\medskip

Donc \fbox{$f'(x) = \dfrac{(2x+1)\sin(x) - (x^2+1)\cos(x)}{(5x-12)^2}$}.
\end{frame}

\section{Remarques}

\subsection{Tableau récapitulatif des opérations}

\begin{frame}{Tableau récapitulatif des opérations}
\begin{center}
\includegraphics[scale=0.9]{tableau_operations.pdf}
\end{center}
\end{frame}

\subsection{Comment calculer une dérivée ?}

\begin{frame}{Comment calculer une dérivée ?}
Lorsqu'on doit calculer la dérivée d'une fonction, la première chose à faire est d'analyser l'expression de la fonction.

\medskip

On doit se demander si 

\begin{itemize}
\item Est-ce une somme ? 
\item Est-ce le produit d'un nombre par un réel ?
\item Est-ce un produit ? 
\item Est-ce un quotient ? 
\item Est-ce une somme de produits ? 
\end{itemize}

\medskip

Puis, on applique les méthodes vues dans cette fiche !

\end{frame}

\end{document}