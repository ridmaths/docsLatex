\documentclass[12pt]{beamer}
\usetheme{Boadilla}
\usecolortheme{seahorse}
\usepackage[utf8]{inputenc}
\usepackage[french]{babel}
\usepackage[T1]{fontenc}
\usepackage{amsmath}
\usepackage{amsfonts}
\usepackage{amssymb}
\usepackage{mathrsfs}
\usepackage{graphicx}
\usepackage{lmodern}
\usepackage{tikz,tkz-tab}

\usepackage{pstricks-add}
%\usepackage{pstricks,pst-plot,pst-text,pst-tree,pstricks-add,pst-grad,pst-coil,pst-blur}

%\usepackage{pgf,pgfplots}
%\pgfplotsset{compat=1.14}
\usetikzlibrary{arrows}

\usefonttheme[onlymath]{serif}

\title[Droites du plan]{Chapitre 12 - Droites du plan}
\author{2GT4}
%\setbeamercovered{transparent} 
%\setbeamertemplate{navigation symbols}{} 
%\logo{} 
%\institute{} 
\date{} 
%\subject{} 

\defbeamertemplate{subsection in toc}{bullets}{%
  \leavevmode
  \parbox[t]{2em}{\phantom{looooll}\hfill}%
  \parbox[t]{1em}{\textbullet\hfill}%
  \parbox[t]{\dimexpr\textwidth-3em\relax}{\inserttocsubsection}\par}
\defbeamertemplate{section in toc}{sections numbered roman}{%
  \leavevmode%
  \MakeUppercase{\romannumeral\inserttocsectionnumber}~-\ %
  \inserttocsection\par}

\setbeamertemplate{section in toc}[sections numbered roman]
\setbeamertemplate{subsection in toc}[bullets]

\makeatletter
\newcommand{\setnextsection}[1]{%
  \setcounter{section}{\numexpr#1-1\relax}%
  \beamer@tocsectionnumber=\numexpr#1-1\relax\space}
\makeatother

\begin{document}

\begin{frame}
\titlepage

\begin{center} \textbf{Partie 2/4} \end{center}
\end{frame}

\begin{frame}
{\Large \textbf{Plan} }

\medskip

\tableofcontents
\end{frame}

\setnextsection{2}

\section{Équation cartésienne d'une droite}

\subsection{Introduction}

\begin{frame}{II - Équation cartésienne}

Dans la partie précédente, nous avons vu la notion de \og vecteur directeur \fg{} pour une droite. 

\bigskip

Une droite peut être caractérisée par deux points ou un point et un vecteur directeur.

\bigskip

Dans cette partie, nous allons voir que les droites peuvent être caractérisées à l'aide d'une équation, appelée \textbf{équation cartésienne}.
\end{frame}

\subsection{Équation cartésienne}

\begin{frame}{Propriété}

\begin{exampleblock}{Propriété}
Dans une repère du plan, les coordonnées $(x;y)$ de tous les points $M$ d'une droite $d$ vérifient une équation de la forme \[ax+by+c = 0\]
où $a$, $b$ et $c$ sont des nombres réels tels que $a$ et $b$ ne sont pas simultanément nuls. \par 
On notera par la suite $(a;b) \neq (0;0)$. \par 
Une telle équation s'appelle une \textbf{équation cartésienne} de la droite $d$.
\end{exampleblock}

\end{frame}

\begin{frame}{Remarques}

\textbf{Remarques : }

\begin{itemize}
\item Une droite $d$ peut être caractérisée par une équation de la forme $ax+by+c = 0$ où $a$, $b$ et $c$ sont des réels avec $(a;b) \neq (0;0)$. \par 
Nous verrons dans la suite comment déterminer une telle équation pour une droite. 

\item Cette équation n'est pas unique : une droite $d$ admet une infinité d'équations cartésiennes. On constate alors que les coefficients sont deux à deux proportionnels.
\end{itemize}

\end{frame}

\begin{frame}
Dans la suite :

\begin{itemize}
\item nous allons voir une démonstration de cette propriété ;
\item nous allons voir comment déterminer une telle équation pour une droite.
\end{itemize}
\end{frame}

\begin{frame}{Démonstration}

Soit $A(x_A ; y_A)$ un point de la droite $d$ et $\overrightarrow{u} \begin{pmatrix} \alpha \\ \beta \end{pmatrix}$ un vecteur directeur de $d$. 

\medskip

Le point $M(x;y)$ appartient à la droite $d$ si, et seulement si, les vecteurs $\overrightarrow{AM}$ et $\overrightarrow{u}$ sont colinéaires. 

\medskip

On a : $\overrightarrow{AM} \begin{pmatrix} x-x_A \\ y-y_A \end{pmatrix}$ et $\overrightarrow{u} \begin{pmatrix} \alpha \\ \beta \end{pmatrix}$.

\medskip

Les vecteurs $\overrightarrow{AM}$ et $\overrightarrow{u}$ sont colinéaires si, et seulement si, $\det(\overrightarrow{AM} ; \overrightarrow{u}) = 0$.

\end{frame}

\begin{frame}{Suite de la démonstration}
Les vecteurs $\overrightarrow{AM}$ et $\overrightarrow{u}$ sont colinéaires si, et seulement si, $\det(\overrightarrow{AM} ; \overrightarrow{u}) = 0$.

\medskip

$M(x;y) \in d$ $\Longleftrightarrow$ $\det(\overrightarrow{AM} ; \overrightarrow{u}) = 0$ \par 
\phantom{$M(x;y) \in d$} $\Longleftrightarrow$ $\beta(x-x_A) - \alpha (y-y_A) = 0$ \par 
\phantom{$M(x;y) \in d$} $\Longleftrightarrow$ $\beta x -\beta x_A - \alpha y + \alpha y_A = 0$ \par 
\phantom{$M(x;y) \in d$} $\Longleftrightarrow$ $\beta x - \alpha y + (-\beta x_A + \alpha y_A) = 0$

\bigskip

On obtient bien une équation de la forme $ax+by+c = 0$ avec $a = \beta$, $b = -\alpha$ et $c = -\beta x_A + \alpha y_A$. \par 
Le vecteur $\overrightarrow{u}$ étant non nul, on a bien : $(a;b) \neq (0;0)$.
\end{frame}

\subsection{Exemple - Méthode - Important}

\begin{frame}{Exemple}

\textbf{Exemple : } soit la droite $d$ passant par $A(1;3)$ et dont un vecteur directeur est $\overrightarrow{u} \begin{pmatrix} 2 \\ 5\end{pmatrix}$. \par 
Déterminer une équation cartésienne de la droite $d$. 

\bigskip

\textit{Réponse (et méthode pour les exercices ) :}

\medskip

Soit $M(x;y)$ appartenant à la droite $d$. 

\medskip

Le vecteur $\overrightarrow{u}$ est un vecteur directeur pour la droite $d$. D'après la définition, les points $A$ et $M$ appartenant à la droite $d$, on sait que les vecteurs $\overrightarrow{AM}$ et $\overrightarrow{u}$ sont colinéaires. 
\end{frame}

\begin{frame}
Or: $\overrightarrow{AM}$ et $\overrightarrow{u}$  sont colinéaires $\Longleftrightarrow$ $\det(\overrightarrow{AM} ; \overrightarrow{u}) = 0$.

\bigskip

$A(1;3)$ et $M(x;y)$ donc : $\overrightarrow{AM} \begin{pmatrix} x-1 \\ y-3 \end{pmatrix}$.

\medskip

On a : $\overrightarrow{AM} \begin{pmatrix} x-1 \\ y-3 \end{pmatrix}$ et $\overrightarrow{u} \begin{pmatrix} 2 \\ 5\end{pmatrix}$.

\medskip

Puis : $\det(\overrightarrow{AM} ; \overrightarrow{u}) = 5(x-1) - 2(y-3)$

\end{frame}

\begin{frame}
$\overrightarrow{AM}$ et $\overrightarrow{u}$  sont colinéaires $\Longleftrightarrow$ $\det(\overrightarrow{AM} ; \overrightarrow{u}) = 0$ \par 
\phantom{$\overrightarrow{AM}$ et $\overrightarrow{u}$  sont colinéaires }$\Longleftrightarrow$ $5(x-1) - 2(y-3) = 0$ \par 
\phantom{$\overrightarrow{AM}$ et $\overrightarrow{u}$  sont colinéaires }$\Longleftrightarrow$ $5x-5 - 2y+6 = 0$ \par
\phantom{$\overrightarrow{AM}$ et $\overrightarrow{u}$  sont colinéaires }$\Longleftrightarrow$ $5x -2y+1 = 0$ \par 

\bigskip

\fbox{Une équation cartésienne de la droite $d$ est $5x-2y+1=0$.}
\end{frame}

\begin{frame}{Résumé de la méthode précédente}

Tout a été détaillé dans l'exemple précédente. Voici un résumé de la méthode à appliquer en exercice :

\bigskip

Comment déterminer une équation cartésienne d'une droite $d$ en partant d'un point $A$ et d'un vecteur directeur $\overrightarrow{u}$ :

\bigskip

\begin{enumerate}
\item Soit $M(x;y)$ appartenant à la droite $d$. On donne les coordonnées de $\overrightarrow{AM}$ en fonction de $x$ et $y$.
\item Le vecteur $\overrightarrow{u}$ est un vecteur directeur de $d$ : les vecteurs $\overrightarrow{u}$ et $\overrightarrow{AM}$ sont colinéaires. 
\item Deux vecteurs sont colinéaires si et seulement si le déterminant est nul : on calcule $\det(\overrightarrow{AM} ; \overrightarrow{u})$ et ce nombre doit être égal à $0$.
\end{enumerate}

\end{frame}

\begin{frame}{Remarque}

\textbf{Remarque : } la méthode précédente nous donne un moyen de déterminer une équation cartésienne pour une droite $d$ connaissant un point $A$ et un vecteur directeur $\overrightarrow{u}$.

\bigskip

Si on a deux points $A$ et $B$ d'une droite $d$, on se rappelle que le vecteur $\overrightarrow{AB}$ est un vecteur directeur de la droite $d$. \par 
Cela nous permet d'obtenir un vecteur directeur pour ensuite appliquer la méthode précédente. 

\end{frame}

\subsection{Propriété réciproque}

\begin{frame}{Propriété réciproque}

\begin{exampleblock}{Propriété}
L'ensemble des points $M$ dont les coordonnées $(x;y)$ vérifient une équation de la forme $ax+by+c=0$, où $a$, $b$ et $c$ sont des réels et $(a;b) \neq (0;0)$, est une droite. 
\end{exampleblock}

\end{frame}

\begin{frame}{Méthode}

\textbf{Pour montrer qu'un point $A(x_A;y_A)$ appartient (ou pas) une droite} :

\bigskip

On considère la droite $d$ dont une équation cartésienne est donné par $ax + by + c = 0$.

\medskip

On sait que : le point $A(x_A ; y_A)$ appartient à la droite $d$ si, et seulement si, $ax_A + by_A + c = 0$. 

\bigskip

Cela donne une méthode pour montrer qu'un point appartient ou n'appartient pas à la droite :

\begin{itemize}
\item Si $ax_A + by_A + c = 0$ alors le point $A$ appartient à la droite $d$.
\item Si $ax_A + by_A + c \neq 0$ alors le point $A$ n'appartient pas à la droite $d$.
\end{itemize}

\end{frame}

\begin{frame}{Exemple}

\textbf{Exemple :} Soit la droite $d$ dont une équation cartésienne est $-2x+3y + 5 = 0$.

\medskip

\begin{enumerate}
\item Le point $A(4;1)$ appartient-il à la droite $d$ ?
\item Le point $B(-2;-2)$ appartient-il à la droite $d$ ?
\end{enumerate}

\end{frame}

\begin{frame}{Corrigé de l'exemple}

\textbf{Exemple :} Soit la droite $d$ dont une équation cartésienne est $-2x+3y + 5 = 0$.

\medskip

\begin{enumerate}
\setcounter{enumi}{0}
\item Le point $A(4;1)$ appartient-il à la droite $d$ ?
\end{enumerate}

On a : 

\medskip

$-2 \times x_A + 3 \times y_A + 5 = -2 \times 4 + 3 \times 1 + 5$ \par 
$\phantom{-2 \times x_A + 3 \times y_A + 5} = -8 + 3 + 5$ \par 
$\phantom{-2 \times x_A + 3 \times y_A + 5} = 0$

\bigskip

Donc $-2 \times x_A + 3 \times y_A + 5 = 0$. 

\medskip

Le point $A$ appartient à la droite $d$.
\end{frame}

\begin{frame}{Corrigé de l'exemple}

\textbf{Exemple :} Soit la droite $d$ dont une équation cartésienne est $-2x+3y + 5 = 0$.

\medskip

\begin{enumerate}
\setcounter{enumi}{1}
\item Le point $B(-2;-2)$ appartient-il à la droite $d$ ?
\end{enumerate}

On a : 

\medskip

$-2 \times x_B + 3 \times y_B + 5 = -2 \times (-2) + 3 \times (-2) + 5$ \par 
$\phantom{-2 \times x_A + 3 \times y_A + 5} = 4 - 6 + 5$ \par 
$\phantom{-2 \times x_A + 3 \times y_A + 5} = 3$

\bigskip

Donc $-2 \times x_B + 3 \times y_B + 5 \neq 0$. 

\medskip

Le point $B$ n'appartient pas à la droite $d$.

\end{frame}

\begin{frame}{Autre exemple}

\textbf{Exemple : } On considère la droite $d$ dont une équation cartésienne est $4x+2y-10 = 0$. Déterminer deux points distincts appartenant à la droite $d$.

\bigskip

\textit{Réponse :} En utilisant l'équation cartésienne, il suffit de choisir $x$ et d'en déduire la valeur de $y$.

\medskip

Pour $x = 0$, on a : $4 \times 0 + 2y - 10 = 0$ donc $2y-10 = 0$. Puis : \par 
$2y-10 = 0 \Longleftrightarrow 2y = 10 \Longleftrightarrow y = 5$.

\medskip

$\bullet$ Le point $A(0;5)$ appartient à la droite $d$.

\medskip

Pour $x = 2$, on a : $4 \times 2 + 2y - 10 = 0$ donc $8 + 2y-10 = 0$. Puis : \par 
$8 + 2y-10 = 0 \Longleftrightarrow 2y-2 = 0 \Longleftrightarrow 2y = 2 \Longleftrightarrow y = 1$.

\medskip

$\bullet$ Le point $B(2;1)$ appartient à la droite $d$.

\end{frame}

\begin{frame}{Remarque}

L'exemple précédent est important car : à partir d'une équation cartésienne pour une droite, on détermine deux points appartenant à la droite.

\bigskip

Si on veut tracer la droite dont une équation cartésienne, il suffit de trouver de cette manière deux points distincts appartenant à la droite, de les placer dans un repère et on obtient notre droite en traçant la droite qui passe par ces deux points. 

\end{frame}

\subsection{Un vecteur directeur}

\begin{frame}{Propriété }

\begin{exampleblock}{Propriété}
On considère une droite $d$ dont une équation cartésienne est donnée par : $ax+by+c = 0$. \par 
Le vecteur $\overrightarrow{u} \begin{pmatrix} -b \\ a \end{pmatrix}$ est un vecteur directeur de~$d$.
\end{exampleblock}

\end{frame}

\begin{frame}{Remarque}

\textbf{Remarque : } La propriété précédente permet, étant donnée une équation cartésienne d'une droite $d$, de déterminer rapidement un vecteur directeur de la droite $d$.

\end{frame}

\begin{frame}{Exemple}

\textbf{Exemple : } On considère la droite $d$ dont une équation cartésienne est $2x-3y-12=0$.

\bigskip

Une équation cartésienne de $d$ est de la forme $ax+by+c=0$ avec $a = 2$ et $b = -3$. 

\bigskip

D'après la propriété précédente, $\overrightarrow{u} \begin{pmatrix} -b \\ a \end{pmatrix}$ est un vecteur directeur de~$d$. 

\medskip

$-b = -(-3) = 3$ et $a = 2$.

\medskip

Donc $\overrightarrow{u} \begin{pmatrix} 3 \\ 2 \end{pmatrix}$ est un vecteur directeur de $d$. 


\end{frame}


\begin{frame}{Autre exemple}

\textbf{Exemple :} Déterminer une équation cartésienne de la droite $d$ passant par le point $A(4;7)$ et dont un vecteur directeur est $\overrightarrow{u} \begin{pmatrix} 5 \\ 11 \end{pmatrix}$.

\bigskip

\textit{Réponse : } d'après la propriété précédente, la droite $d$ admet une équation de la forme $ax+by+c = 0$ ou le vecteur $\overrightarrow{u} \begin{pmatrix} -b \\ a \end{pmatrix}$ est un vecteur directeur de $d$. 

\medskip

Donc $-b = 5$ et $a = 11$. Ce qui donne $a=11$ et $b = -5$.

\end{frame}

\begin{frame}
Une équation cartésienne de la droite $d$ est $11x-5y+c = 0$.

\bigskip

Pour trouver la valeur $c$, nous allons utiliser le point $A$

\bigskip

$A(\textcolor{blue}{4};\textcolor{red}{7}) \in d \Longleftrightarrow 11 \times \textcolor{blue}{4} + (-5) \times \textcolor{red}{7} + c = 0$ \par  
$\phantom{A(\textcolor{blue}{4};\textcolor{red}{7}) \in d} \Longleftrightarrow 44 -35 + c = 0$ \par 
$\phantom{A(\textcolor{blue}{4};\textcolor{red}{7}) \in d} \Longleftrightarrow 9 + c = 0$ \par 
$\phantom{A(\textcolor{blue}{4};\textcolor{red}{7}) \in d} \Longleftrightarrow c = -9$.


\bigskip

Une équation cartésienne de la droite $d$ est $11x-5y - 9 =0 $.

\end{frame}

\begin{frame}
\begin{center}
\textbf{Fin de la partie 2}
\end{center}

Feuille d'exercices 2 à traiter !
\end{frame}

\end{document}