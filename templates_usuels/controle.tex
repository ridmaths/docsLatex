\documentclass[11pt,a4paper]{article}

% Packages Ecriture
\usepackage[T1]{fontenc}
\usepackage[utf8]{inputenc}
\usepackage{fourier}
\usepackage[scaled=0.875]{helvet} 
\renewcommand{\ttdefault}{lmtt} 
\usepackage{eurosym}
\usepackage[frenchb]{babel}
\usepackage[np]{numprint}

% Packages Mise en page
\usepackage{geometry} % Permet la mise en page
\geometry{lmargin=1.5cm,rmargin=1.5cm,tmargin=1.5cm,bmargin=1cm} % Choix format et  des marges
\usepackage{fancyhdr}
\usepackage{framed} % Création d'environnements
\usepackage[framed,thmmarks]{ntheorem} % Gestions des environnements theorem
\usepackage{lastpage}

\usepackage{amsmath,amssymb,makeidx}
%\usepackage{enumerate}
\usepackage{enumitem}
\usepackage[normalem]{ulem}
\usepackage{fancybox,graphicx}
\usepackage{tabularx}
\usepackage{ulem}
\usepackage{dcolumn}
\usepackage{textcomp}
\usepackage{diagbox}
\usepackage{tabularx}
\usepackage{lscape}
\usepackage{pstricks,pst-plot,pst-text,pst-tree,pstricks-add,pst-grad,pst-coil,pst-blur}
\usepackage[pstricks]{bclogo} % Logo
%\setlength\paperheight{297mm}
%\setlength\paperwidth{210mm}
%\setlength{\textheight}{25cm}

\usepackage{hyperref}

%\thispagestyle{empty}
%\usepackage[frenchb]{babel}
%\usepackage[np]{numprint}

\usepackage{pgf,tikz,tkz-tab}
\usepackage{mathrsfs}
\usetikzlibrary{arrows}

%\usepackage{showframe}


%%%%%%%%%%%%%%%%%%%%%%%%%%%%%%%%%%%%%%%%%%%%%%%%%%%%%%%%%%%%%%%%%%%%%%%%%%%%%%%%%%%%%%%%%%%%%%%%
%%%%%%%%%%%%%%%%%%%%%%%%%%%%%%%%%%%%%%%%%%%%%%%%%%%%%%%%%%%%%%%%%%%%%%%%%%%%%%%%%%%%%%%%%%%%%%%%

\DecimalMathComma	% Définit la virgule comme séparateur entre la partie entière et la partie décimale


\def\Oij{$\left(\textrm{O},~\vec{\imath},~\vec{\jmath}\right)$}
\def\Oijk{$\left(\textrm{O},~\vec{\imath},~\vec{\jmath},~\vec{k}\right)$}
\def\Ouv{$\left(\textrm{O},~\vec{u},~\vec{v}\right)$}

\newcommand{\vect}[1]{\mathchoice%
{\overrightarrow{\displaystyle\mathstrut#1\,\,}}%
{\overrightarrow{\textstyle\mathstrut#1\,\,}}%
{\overrightarrow{\scriptstyle\mathstrut#1\,\,}}%
{\overrightarrow{\scriptscriptstyle\mathstrut#1\,\,}}}

\newcommand{\equi}{\Leftrightarrow}
\newcommand{\pg}{\geqslant}
\newcommand{\pp}{\leqslant}
\newcommand{\dt}{\,\mathrm{d}t}
\newcommand{\dx}{\,\mathrm{d}x}

\newcommand{\N}{\mbox{${\mathbb N}$}}
\newcommand{\Z}{\mbox{${\mathbb Z}$}}
\newcommand{\Q}{\mbox{${\mathbb Q}$}}
\newcommand{\R}{\mbox{${\mathbb R}$}}
\newcommand{\C}{\mbox{${\mathbb C}$}}

%Commande pour créer des lignes de pointillés
\newcommand{\Pointilles}[1][3]{%
\multido{}{#1}{\makebox[\linewidth]{\dotfill}\\[\parskip]
}}

\DeclareMathOperator{\e}{e} %Permet d'écrire "droit" le e de l'exponentielle

\let\oldarray=\array
\def\array{\small\oldarray} % Permet d'écrire plus petit dans le tableau

\renewcommand{\arraystretch}{1.2} % Permet de gérer la hauteur des lignes des tableaux

\parindent=0mm % Supprime l'alinéa

\setlength{\fboxsep}{2mm} % Gère l'espacement entre un cadre et son contenu

\renewcommand{\thesection}{\Roman{section}.} % Numérote les sections en chiffre romain
\renewcommand{\thesubsection}{\Alph{subsection}.}


\newcounter{nexo}
\setcounter{nexo}{0}
\newcommand{\exo}{%
\stepcounter{nexo}
{\textbf{\underline{\textsc{Exercice \arabic{nexo}}}}%
\bigskip%
}
}

\pagestyle{empty}

\begin{document}

%\thispagestyle{empty}

\setlength\parindent{0mm}

~\vspace{-1.5cm}

\begin{center} \huge{\framebox{\textbf{Contrôle n°1 de Mathématiques - 2GT4}}} \end{center}

\begin{center} \Large{Lundi 7 octobre 2019} \end{center}

\medskip

Nom : \dotfill Prénom : \dotfill

\medskip

\textit{Toutes les réponses devront être justifiées.} \textit{Le barème est donné à titre indicatif.} \par 
\textbf{Le sujet est recto-verso ! NE PAS OUBLIER DE RETOURNER LA FEUILLE !!} \par 
\textbf{Le sujet est à rendre avec la copie.}

\bigskip

\exo \hfill (5 points)

Sur la figure donnée ci-dessous, on considère le parallélogramme $ABCD$. \par 
Le point $E$ est l'image du point $D$ par la translation de vecteur $\overrightarrow{BA}$. \par 
Soit le point $F$ tel que le point $D$ soit l'image du point $F$ par la translation de vecteur $\overrightarrow{EA}$. \par 

\begin{center}
\includegraphics[scale=0.6]{fig1.pdf}
\end{center}

\begin{enumerate}
\item Placer les points $E$ et $F$ sur la figure ci-dessus.
\item Justifier que : $\overrightarrow{AB} = \overrightarrow{ED}$. \par 
Que peut-on en déduire de cette égalité ?
\item Justifier que : $\overrightarrow{AB} = \overrightarrow{DC}$.
\item À l'aide des questions 2 et 3, que peut-on dire du point $D$ ? Justifier.
\item Justifier la nature du quadrilatère $FEBC$.
\end{enumerate}

\bigskip

\exo \hfill (5 points)

On considère la triangle $ABC$ donné ci-dessous.

\begin{center}
\includegraphics[scale=0.6]{fig2.pdf}
\end{center}

\begin{enumerate}
\item Construire les points $I$, $J$ et $K$ définis par : 

\medskip

\text{a)~~} $\overrightarrow{AI} = \overrightarrow{AB} + \overrightarrow{AC}$ ; 

\medskip

\text{b)~~} $\overrightarrow{AJ} = \overrightarrow{AB} - \overrightarrow{AC}$ ; 

\medskip

\text{c)~~} $\overrightarrow{AK} = 2\overrightarrow{AB} - \overrightarrow{AC}$.

\item Exprimer le vecteur $\overrightarrow{JA}$ en fonction des vecteurs $\overrightarrow{AB}$ et $\overrightarrow{AC}$.
\item En utilisant la relation de Chasles, démontrer que $\overrightarrow{JK} = \overrightarrow{AB}$.
\item Démontrer que $\overrightarrow{CI} = \overrightarrow{AB}$.
\item Quelle est la nature du quadrilatère $CIKJ$ ? Justifier.
\end{enumerate}

\bigskip

\vfill

\begin{center} \textbf{Page 1 sur 2} \end{center}

\newpage

\exo \hfill (4 points)

Pour chacun des nombres suivants, indiquer, en justifiant, le plus petit ensemble de nombres auquel il appartient. 

\medskip

\qquad \textbf{a)~~} $9$ ; \quad \textbf{b)~~} $\dfrac{13}{8}$ ; \quad \textbf{c)~~} $-\dfrac{5}{3}$ ; \quad \textbf{d)~~} $-\dfrac{72}{6}$ ; 

\medskip

\qquad \textbf{e)~~} $(-\sqrt{3})^6$ ; \quad \textbf{f)~~} $\sqrt{2 \times ((\sqrt{2})^2-1)}$.

\bigskip

\bigskip

\exo \hfill (4 points)

Écrire, \textbf{en détaillant}, chaque expression sous la forme $a^n$, où $a$ et $n$ sont des nombres relatifs :

\medskip

\qquad \textbf{a)~~} $A = 5^6 \times 5^{12}$ ; \quad \textbf{b)~~} $B = 11^{-4} \times (11^3)^5$ ; \quad \textbf{c)~~} $C = 5^{10} \times 7^{10}$ ; \par

\medskip

\qquad \textbf{d)~~} $D = \dfrac{70^9}{14^9}$ ; \quad \textbf{e)~~} $E = \dfrac{3^5}{3^6 \times 3^{14}}$ ; \quad \textbf{f)~~} $F = \dfrac{13^7 \times 13^3}{13^2 \times 13^4}$.

\bigskip

\bigskip

\exo \hfill (3 points)

Écrire, \textbf{en détaillant}, chaque expression sous la forme $\sqrt{a}$ avec $a > 0$.

\medskip

\qquad \textbf{a)~~} $A = \sqrt{3} \times \sqrt{5}$ ; \quad \textbf{b)~~} $B = \dfrac{\sqrt{45}}{\sqrt{9}}$ ; \quad \textbf{c)~~} $C = \sqrt{49} + \sqrt{25}$ ; \quad \textbf{d)~~} $D= 7\sqrt{3}$.

\bigskip

\bigskip

\exo \hfill (4 points)

Écrire, \textbf{en détaillant}, chaque expression sous la forme $a \sqrt{b}$, avec $a$ et $b$ entiers et $b$ étant le plus petit possible :

\medskip

\qquad \textbf{a)~~} $A = \sqrt{80}$ ; \quad \textbf{b)~~} $B = 6\sqrt{12}$ ; \quad \textbf{c)~~} $C = \sqrt{72} + 2\sqrt{128}$ ;  \quad \textbf{d)~~} $D = 2\sqrt{448} + 3\sqrt{1183} + \sqrt{1575}$.

\vfill

\begin{center} \textbf{Page 2 sur 2} \end{center}

\end{document}

\exo \hfill (4 points)

Pour chacun des nombres suivants, indiquer, en justifiant, le plus petit ensemble de nombres auquel il appartient. 

\medskip

\qquad \textbf{a)~~} $9$ ; \quad \textbf{b)~~} $\dfrac{13}{8}$ ; \quad \textbf{c)~~} $-\dfrac{5}{3}$ ; \quad \textbf{d)~~} $-\dfrac{72}{6}$ ; \quad \textbf{e)~~} $(-\sqrt{3})^6$ ; \quad \textbf{f)~~} $\dfrac{\sqrt{128}}{\sqrt{2}}$ ; \quad \textbf{g)~~} $\sqrt{2 \times ((\sqrt{2})^2-1)}$.

\bigskip

\bigskip

\exo \hfill (4 points)

Écrire, \textbf{en détaillant}, chaque expression sous la forme $a^n$, où $a$ et $n$ sont des nombres relatifs :

\medskip

\qquad \textbf{a)~~} $A = 5^6 \times 5^{12}$ ; \quad \textbf{b)~~} $B = \dfrac{4^3}{4^{11}}$ ; \quad \textbf{c)~~} $C = 11^{-4} \times (11^3)^5$ ; \quad \textbf{d)~~} $D= 5^{10} \times 7^{10}$ ; \par

\medskip

\qquad \textbf{e)~~} $E= \dfrac{70^9}{14^9}$ ; \quad \textbf{f)~~} $F = \dfrac{3^5}{3^6 \times 3^{14}}$ ; \quad \textbf{g)~~} $G = \dfrac{13^7 \times 13^3}{13^2 \times 13^4}$.

\bigskip

\bigskip

\exo \hfill (3 points)

Écrire, \textbf{en détaillant}, chaque expression sous la forme $\sqrt{a}$ avec $a > 0$.

\medskip

\textbf{a)~~} $A = \sqrt{3} \times \sqrt{5}$ ; \quad \textbf{b)~~} $B = \dfrac{\sqrt{45}}{\sqrt{9}}$ ; \quad \textbf{c)~~} $C = \sqrt{49} + \sqrt{25}$ ; \quad \textbf{d)~~} $D= 7\sqrt{3}$.

\bigskip

\bigskip

\exo \hfill (4 points)

Écrire, \textbf{en détaillant}, chaque expression sous la forme $a \sqrt{b}$, avec $a$ et $b$ entiers et $b$ étant le plus petit possible :

\medskip

\qquad \textbf{a)~~} $A = \sqrt{80}$ ; \quad \textbf{c)~~} $C = 6\sqrt{12}$ ; \quad \textbf{d)~~} $D = \sqrt{28} + \sqrt{112}$ ; \par 

\medskip

\qquad \textbf{e)~~} $E= 3\sqrt{150} - 2\sqrt{96}$ ; \quad \textbf{f)~~} $F= \sqrt{448} + \sqrt{1183} + \sqrt{1575}$.


\textbf{BONUS :}

On admet que $\sqrt{2}$ est un nombre irrationnel. Montrer que le nombre $5\sqrt{2}$ est aussi un nombre irrationnel. 

\bigskip