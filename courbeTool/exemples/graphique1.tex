\documentclass{standalone}
\usepackage[T1]{fontenc}
\usepackage[utf8]{inputenc}
\usepackage{fourier}
\usepackage[scaled=0.875]{helvet} 
\renewcommand{\ttdefault}{lmtt} 
\usepackage{eurosym}
\usepackage[french]{babel}
\usepackage[np]{numprint}
\usepackage{amsmath,amssymb,mathrsfs}
\usepackage{pstricks,pst-plot,pstricks-add}

\begin{document}

%% Configuration
\psset{algebraic=true,dimen=middle,dotstyle=o,dotsize=5pt 0,linewidth=1.6pt,arrowsize=3pt 2,arrowinset=0.25}

\begin{pspicture*}(-5,-5)(5,5)

%% Grille
% Lignes horizontales
\multips(0,-5)(0,1){11}{\psline[linestyle=dashed,linecap=1,dash=1.5pt 1.5pt,linewidth=1pt,linecolor=lightgray]{c-c}(-5,0)(5,0)}
% Lignes verticales
\multips(-5,0)(1,0){11}{\psline[linestyle=dashed,linecap=1,dash=1.5pt 1.5pt,linewidth=1pt,linecolor=lightgray]{c-c}(0,-5)(0,5)}

%% Axes
\psaxes[labelFontSize=\scriptstyle,xAxis=true,yAxis=true,Dx=1.,Dy=1.,ticksize=-2pt 0,subticks=1]{->}(0,0)(-5,-5)(5,5)

%% Trac� des fonctions
% Fonction
\psplot[linewidth=2.pt,linecolor=blue]{0}{6}{(2/3)*x+1}
\rput[bl](3.5,-1.5){\blue{$d$}}
% Fonction
\psplot[linewidth=2.pt,linecolor=red]{0}{6}{-1.5*x+3.5}
\rput[bl](3.5,-1.5){\red{$d'$}}
% Fonction
\psplot[linewidth=2.pt,linecolor=green]{0}{6}{x^2}
\end{pspicture*}
\end{document}